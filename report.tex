\documentclass[UTF8]{ctexart}
\usepackage{geometry}
\usepackage{graphicx}
\usepackage{ulem}
\usepackage{blindtext}
\usepackage{index}
\usepackage{fancyhdr}
\usepackage[colorlinks,linkcolor=black]{hyperref}
\geometry{a4paper, centering, scale=0.75}
\newenvironment{intro}{\begin{quote}\kaishu}{\end{quote}}

\usepackage{fancyhdr}                                
\usepackage{lastpage}                                           
\usepackage{layout}                                 
\ctexset{
    section = {
        format = \raggedright\Large\bfseries,
    }
}

%\pagestyle{empty}                  %不设置页眉页脚            
\footskip = 10pt                                                
\pagestyle{fancy}                   % 设置页眉                                                
% \lhead{}                    
\chead{毛泽东思想和中国特色社会主义理论体系概论实践报告}                                                
\rhead{}                                                
\cfoot{第 \thepage\ 页\quad 共 \pageref{LastPage} 页}                                                
% \rfoot{页脚左}%                                                       
% \lfoot{页脚右}                                                        
\renewcommand{\headrulewidth}{1pt} %页眉线宽,设为0可以去页眉线
\setlength{\skip\footins}{0.5cm}   %脚注与正文的距离           
\renewcommand{\footnotesize}{}     %设置脚注字体大小           
\renewcommand{\footrulewidth}{1pt} %脚注线的宽度           

\begin{document}
\fancypagestyle{plain}
\fancyhf{}
\begin{titlepage}
    \vspace*{\fill}
    \begin{figure}[h]
        \centering
        \includegraphics[width=9cm]{imgs/logo.jpg}
    \end{figure}
    \vskip 15mm
    \begin{center}
        \normalfont
        {\Large\bfseries 毛泽东思想和中国特色社会主义理论体系概论实践报告}
        
        \vskip 5mm
        \normalfont{\Huge\bfseries 瞻前与顾后}

        \vskip 30mm
        {\Large\itshape 丁优龙}{\Large 2019151088}

        \medskip
        {\Large\itshape 费楚雯}{\Large 2019191139}
        
        \medskip
        {\Large\itshape 董芸豪}{\Large 2019284073}
        
        \medskip
        {\Large\itshape 霍晓雨}{\Large 2019151092}
    \end{center}
    \vspace{\stretch{0}}
\end{titlepage}
\clearpage
\section*{\Large 实践主题}
探访文化之根、促进中外交流\raisebox{0.5mm}{------}中外友人南头古城之旅
\section*{\Large 实践目的}
为加深对传统文化历史的了解,加强国际文化交流,粤桂社区党委联合深圳大学国际交流学院各外籍学生,以及其他学院的中国志愿者,组织了一次别开生面的文化交流活动,共同前往深圳文化之根南头古城。
\section*{\Large 实践时间}
2020年11月6日下午2:00---5:00
\section*{\Large 实践地点}
深圳市南山区南头古城
\section*{\Large 实践形式}
在南头古城导游的带领下,随同外国友人游览南头古城风貌并承担同步翻译的任务,参观1820数字展馆及同源馆,品尝南头古城特色美食,进行友好互动及交流。本次探访交流之旅有来自中国、巴基斯坦、美国、日本、韩国、越南、柬埔寨、俄罗斯、印度共9个国家的学生,各位学生互相沟通,互相碰撞。
\section*{\Large 正文}

\end{document}